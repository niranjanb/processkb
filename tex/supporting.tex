% !TEX root =  main.tex
\section{Development Plan and Timeline}

The project will proceed in four stages. 1) Representation design 2) Extraction methods 3) Expansion methods 4) Curation and release. The developed components will be evaluated and improved continually. The research plan in calendar years is shown below:\\
{\bf Year 1: }
\begin{enumerate}[noitemsep,nolistsep]
\item Inferential needs gathering and representation design.
\item Implement sentence gathering. 
\item Extraction and joint inference methods.
\item Intrinsic evaluation of the knowledge base.
\end{enumerate}
{\bf Year 2:}
\begin{enumerate}[noitemsep,nolistsep]
\item Investigate methods for improving joint inference and iterative expansion.
\item Expansion of the KB to all target concepts and evaluation of the resulting knowledge base and 
\item Curation and release of knowledge base. Open source release of the software and web service.
\end{enumerate}
\vspace{-1em}

\section{Broader Impacts of the Proposed Activities}

Knowledge about processes is fundamental to our understanding of the world and vital for AI systems that interact with the world and with humans.  Computational representations of scientific knowledge has tremendous applications in improving access to critical information, as well as in accelerating discovery and research processes. Moreover AI systems are increasingly adopted for use in many types of decision making in critical areas such as technology, science, and medicine. Knowledge based reasoning is crucial for building systems that have the ability to explain their decisions or predictions. Such systems require an understanding how to represent knowledge in a form that lends itself to computational reasoning for complex tasks.

\subsection{Curriculum Development Activities}

The PI teaches grad-level Introduction to Natural Language Processing, and Advanced Topics in Computational Linguistics. 
With advent of big data ecosystem understanding and extracting knowledge is becoming increasingly relevant in industry.
I plan to teach a course centered around the core concepts of knowledge representation, and scalable extraction techniques for knowledge. This course is relevant for both Masters and PhD students. Most NLP-based technology companies and technology companies with a large web presence have a need for extracting and organizing knowledge from their user engagement data. This course will provide a basic overview of a distributed information extraction pipeline, persistence, and building applications that rely on the extracted data. 

\subsection{Community Outreach}

There is an increased enthusiasm for advanced placement computer science courses in the high schools in local communities. 
The proposed project deals with computational representations of scientific processes discussed in grade level sciences. 
This provides an unique opportunity to introduce computer science concepts using an application domain that they are familiar with.
The familiarity and the far reaching impact possibilities provide an excellent platform to attract the attention of high school students.
The PI plans to offer introductory technical lectures, and project opportunities to engage high school students in the context of the proposed project. 

\section{Prior NSF Support}

Dr. Niranjan Balasubramanian has broad expertise in information extraction, esp. in large scale knowledge generation and its application to complex NLP tasks but has not received prior support from NSF. 
