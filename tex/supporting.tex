% !TEX root =  main.tex
\section{Development Plan and Timeline}

The project will proceed in three phases. In the first phase, we will design the representation and methods for role discovery and acquisition. In the second phase, we will start curating the generated knowledge and build a semantic role labeler for questions. In the third phase we will refine and make necessary adjustments to the pipeline, finish the curation and release the resources. The research plan in calendar years is shown below:


Year 1:  
\begin{enumerate}[noitemsep,nolistsep]
\item 
\item 
\item 
\end{enumerate}
Year 2:
\begin{enumerate}[noitemsep,nolistsep]
\item 
\item 
\item 
\end{enumerate}


\section{Broader Impact}


\section{Curriculum Development Activities}

I plan to teach a course centered around the core concepts of knowledge representation, and scalable extraction techniques for knowledge. This course is relevant for both Masters and PhD students. Most NLP-based technology companies and technology companies with a large web presence have a need for extracting and organizing knowledge from their user engagement data. This course will provide a basic overview of a distributed information extraction pipeline, persistence, and building applications that rely on the extracted data. 

\section{Prior NSF Support}

Dr. Niranjan Balasubramanian has broad expertise in information extraction, esp. in large scale knowledge generation and its application to complex NLP tasks but has not received prior support from NSF. 
