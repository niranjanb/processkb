% !TEX root =  main.tex
\section{Role Assessment and Discovery}

Inference yields a set of role fillers that can be reliably identified from the input set of sentences. 
The yield is limited by the set of roles and the query patterns used to retrieve sentences for these roles.

{\bf Assessment}

To expand the knowledge further, we propose an iterative procedure that learns from the inferred roles to find new query patterns. 
Traditional bootstrapping methods coupled with query expansion techniques from information retrieval 
can be exploited to instantiate query templates with new patterns derived from the retrieved sentences. 
The new query patterns are then used to repeat the entire procedure to derive new fillers. 

The iterative procedure however can yield role fillers that are inconsistent with roles obtained from previous iterations.
While it is possible to infer new roles jointly with the current KB at each iteration, it can introduce too many variables in inference and render it inefficient.
Therefore, we propose a simple aggregation procedure that consolidates the roles by effectively resolving any inconsistencies between the different iterations.
%Last we will also explore methods that automatically assesses the generated knowledge in terms of coverage and quality

{\bf Role Discovery and Refinement}

At each iteration we will also inspect if there are any new roles that need to be added to the role set.
For many processes there are specific important roles that do not fit any of the general roles. Some examples:
	\begin{itemize}
		\item {\em Phototropism} is the mechanism by which plants grow towards a light source. 
		The notion of a target {\em light source} and the {\em direction} or {\em orientation} of the growth
		are critical for distinguishing positive and negative phototropism. However, neither notion fits
		with any of the existing roles. 
	\item {\em Heat transfer} processes such as radiation have notion of a {\em medium}, 
		which is critical for distinguishing between instances such as convection and radiation
		Again medium doesn't fit with any existing roles. 
	\end{itemize}

Our objective is similar to the goals of prior work on unsupervised role induction. They use a deterministic procedure for candidate argument identification and clustered syntactic signatures of these arguments to induce roles. However, unlike the standard unsupervised setting, we have a set of roles that have been identified already. Further, we find that most of these process specific new roles tend to be realized via prepositional, noun-noun, or other noun modifier relations that attach to one of the existing roles. Information about how these candidates are related to currently identified roles is likely to help.

\begin{itemize}
\item We propose to investigate methods for breaking coarse-grained roles into multiple sub-roles.
 and preposition relation extraction tools 

We use two sources for identifying candidates. 
First, we obtain candidates from the local extraction pipeline that were assigned low scores by the inference and then we extend it with candidates from a separate PropBank style argument identification pipeline~\cite{}. Second, we also inspect role fillers for existing roles that can be broken up into fine-grained roles. For example, "towards a light source" can be further split into two roles one relating to the "direction" of the growth and the other relating to the goal "light source".

\item Having identified candidates we iterate through the pipeline to find additional sentences that contain these candidates and the core roles or predicates for the process. Following prior role induction work, we extract a context signature for each candidate and cluster the candidates that are realized with similar contexts. As mentioned earlier, we propose to also use the semantic role context of the candidates. 

\end{itemize}







