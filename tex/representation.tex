% !TEX root =  main.tex
\section{Representation}
\label{sec:schemas}

%{\bf What are the guiding principles for the representation?}
Our goal is to design a representation that meets the inferential needs in downstream applications. 
While we cannot anticipate all needs in advance, we turn to the the grade-level process questions for determining a role vocabulary.
Our prior work showed that general purpose semantic roles, and a handful of domain-specific 
roles provide effective coverage for a majority of the recognition questions. 
We will extend these with roles from existing resources such as FrameNet and PropBank to construct a comprehensive role vocabulary.

%{\bf What are the roles to use? Adopt PropBank roles but define it for the process -- not at the verb level.}
Table~\ref{tab:roles} lists a standard list of semantic roles~\cite{}. These roles are typically defined with respect to a predicate, most often a verb.
However, there is substantial variability in discourse reflecting the diversity in situations involving the same processes. 
Consider the following sentences:
\begin{itemize}
\item When water evaporates it changes to water vapor. 
\item The process by which water from the oceans rises into the atmosphere as water vapor. 
\end{itemize}

Both sentences describe instances of the process of evaporation. 
With respect to the corresponding predicates, Water vapor is the {\em Goal} in the first sentence, but is a {\em Patient} in the second. 
However, both are essentially the {\em Result} of the process evaporation. 

This fragmentation is problematic for both interpreting situations and for drawing inferences about them. 
To avoid these problems we propose to define the roles with respect to a canonical description of the process. 
Further we aim to assemble sentences that express information in expected ways, thus allowing us 
flexibility in determining canonical lexical realizations.
%As we will detail in the next section, we aim to assemble sentences that express information in expected ways 
%thereby addressing the difficulties of dealing with sentences that express information in non-canonical ways. 

\begin{table}[htdp]
\caption{Semantic Role Vocabulary}
\begin{center}
\begin{tabular}{|l|l|}
\hline
agent & Causal agent of an event.\\
instrument & The object used to accomplish the goal.\\
cause & Event or Agent that causes the event to happen.\\
experiencer & The entity undergoing the event. \\
recipient & The benefactor of the process outcome.\\
path & The path through which the process happens.\\
location & The place where the process happens.\\
measure & A quantity associated with the process.\\
theme & A process entity that doesn't undergo any change.\\
\hline

\end{tabular}
\end{center}
\label{tab:roles}
\end{table}%

%{\bf Why is discovering necessary?}
Manual assignment of roles for every process is not scalable. 
Instead we propose to automatically determine the applicable subset of roles from this vocabulary. 
%Our key idea is that some subset of roles can be identified with higher precision than others. 
%For instance roles such as {\em cause} and {\em result} have highly regular lexical realizations.
%{\bf What is the method for discovering?}
%{\bf Why will this work?}
%{\bf What work exists? How does this relate to x?}