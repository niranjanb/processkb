eat{\section*{Notes}
\begin{itemize}
\item Are you pitching the resource or the technique? The answer is the technique. The pitch is essentially that of acquiring representations about any target resource by explicitly querying for easy to extract sentences. This works for certain types of target concepts.
\item Pitch broadly to any domain where the target phenomena are described in many ways. 
\item Pitch utility for other applications.
\item How is this different from relation extraction?
\item Acknowledge the focus is on sentences that contain the process name. Often times discourse is needed to extract other information.
\item What is the secret sauce? It is the sauce that was used for Open IE, ie., leveraging extractable sentences those which convey information in expected ways.
\item What is the formalism? Constrained Conditional Models or Graphical Models? 
\item Is there any issue with using Google API? Mention you will consider using ClueWeb or PeterT's corpus.
\item How is the work different from semi-supervised and unsupervised semantic role induction?
\item How is this different from template induction for event extraction? 
\end{itemize}
\newpage
}

Angles:

1) Extraction 
	-- What is a clear contribution in extraction that is missing in existing systems?
	-- Is the output new? If so, what changes are needed to existing systems to arrive at this output.
	-- If the output is new, then you have to motivate why this output is helpful. 
	
2) QA with SRL and Frame-Semantic based systems 
	-- Is there a new attempt here? Previous QA systems did just matching? 

Are definitions somehow fundamentally different from instances?
The big problem is still entailment -- representation and sense issues are the key? 
4th grade QA is different from open-domain or terrorist domain QA in that it requires more understanding? 
Are process recognition questions unique in some respect, compared to general purpose QA? 
	-- Yes, in the sense that we don't have to do other types of reasoning (qualitative comparison etc.)

Build on SRL and Frame-Semantic Parsing systems 
	-- Don't invent a new formulation. Pick one.
	
	

The most exciting narrative is where the output is new. That is we are reading information about processes into a structured representation, that goes beyond processing single sentences. In particular we are reading inferential knowledge. 

input(x) is-a liquid
output(x) is-a vapor
enabler(x) is-a heat-source
	=> evaporation
	
Some of this falls into deductive reasoning and some into abductive.

	