%!TEX root = summary-only.tex
\begin{center} 
	{\bf CRII:III: Composing Process Knowledge using Semantic Roles}\\
	{Niranjan Balasubramanian, Stony Brook University}

 \end{center}

Large-scale text resources and fact databases have spurred significant advances 
in AI systems that answer factual questions  (e.g., "When was Bill Clinton born?").
To address more challenging problems that go beyond factual answer retrieval,
AI systems need to be able to reason about general scenarios. 
They need to know general truths or generalities and how to apply them to specific situations. 
Acquiring this type of general knowledge and using it to reason effectively remains a huge challenge.

We explore this challenge for reasoning about {\em processes} in order to answer
questions about specific scenarios involving them. 
In particular we propose to compose knowledge about processes in terms of semantic roles (e.g., What is undergoing the process? What is the result? etc.). We develop solutions to automatically construct a large repository of simple process knowledge, and demonstrate its use to answer process questions that go beyond fact lookup. 

Existing lexical semantic resources provide similar forms of semantic knowledge about general open-domain actions but lack coverage on scientific processes. FrameNet, for instance, does not have entries for nearly half of the processes described in 4th grade science exams. The coverage is likely worse for higher grade levels with deeper knowledge domains. 

%. For example, PropBank and FrameNet provide role knowledge about open-domain actions and have been successfully used in information extraction and open-domain QA. While they provide exhaustive coverage for open-domain actions (verbs), their coverage on processes in the science domain is lacking. FrameNet, for instance, does not have entries for nearly half of the processes described in 4th grade science exams. The coverage is likely worse for higher grade levels with deeper knowledge domains. 
In response we propose to investigate methods to automatically construct a large repository of simple semantic role based knowledge about processes. Adapting existing semantic role labeling systems to work well on new domains is difficult. We take a different approach. Our key premise is that rather than building a semantic role labeler that works well on any sentence, we proactively gather sentences that convey information in expected constructs. We propose a framework that combines techniques from extraction and joint inference for finding semantic roles and iteratively expands its knowledge to discover roles on its own.
Specifically, we will make three main contributions:
\begin{itemize}[noitemsep,nolistsep]
\item The first comprehensive, large-scale {\bf knowledge base of processes} in the grade science domain, describing the roles
      and changes involved in that process.
\item Methods for {\bf automatic extraction of process knowledge} using information extraction and joint inference.
\item A framework for {\bf iterative knowledge expansion}, which allows the system to discover new roles
     involved in a process and expand the process representation to accommodate them.
\end{itemize}

{\bf Intellectual Merit} 

This work investigates a new direction in acquiring semantic roles by targeting sentences that express information in expected ways. The work will lead to better understanding and advancement of cross-sentence alignment of semantic roles and collective labeling. This work will push understanding on continuous learning (similar to the NELL project for relation extraction) with the iterative expansion methods and increase understanding of the connection between regularities of syntactic realizations and semantic roles. Overall the project also contributes to advances in leveraging unlabeled data and methods for balancing representational needs against extraction capabilities.

{\bf Broader Impact}

Automatic knowledge extraction is fundamental to advancing Artificial Intelligence. 
This work contributes towards building systems capable of understanding knowledge in texts and reason with them.
Better reasoning systems help information access and reduce information overload thereby accelerating research and discovery, 
as well as serve the information needs of the population at large.

{\bf Keywords} Knowledge extraction, Semantic Role Labeling, Information Extraction, Question Answering


% -- In order to scale, we will investigate methods for automatically determining which roles are applicable for a particular process. Prior works have either relied on fully supervised training data to specify roles~\cite{} or have relied on clustering of syntactic functions to automatically induce roles~\cite{}. We propose an intermediary solution that aims to determine which set of roles (from a pre-specified role vocabulary) applies to the process. The intuition is that there are some general purpose notions such as {\em direction}, {\em medium}, {\em change} etc that are common to many processes.
%	
\eat{Knowledge-intensive benchmarks including standardized grade-level science exams~\cite{clark2015elementary}, 
textual entailment~\cite{dagan2010recognizing}, and reading comprehension tasks~\cite{richardson2013mctest} have renewed interests in automatic knowledge extraction and reasoning. 
These tasks motivate reasoning-based systems that go beyond simple retrieval and syntactic structure matching~\cite{clark2014:akbc,chb2013:akbc, khot2015:emlnlp}. 

Effective semantic representations of relevant information is critical for making significant advances in building such systems. 
Semantic role-based representations have shown promise for open-domain factoid question answering~\cite{shen2007using, pizzato2008indexing}. 
Some of the knowledge required for grade-level science exams are naturally expressed via semantic roles.
Consider the following example from an actual 4th grade science exam.

{\em When plants use stored sugar for energy, they go through a process called \\
(A) photosynthesis (B) transpiration (C) respiration (D) perspiration.}\\

{\em Photosynthesis} and {\em respiration} both involve sugar and energy. 
Photosynthesis converts light energy to sugar, whereas (cellular) respiration releases energy in the sugar by breaking it down. 
Not surprisingly these processes are described using similar words, which makes bag-of-words style reasoning unreliable. 
Understanding the different roles that energy and sugar play in these processes is key to effective reasoning.
Such semantic roles are critical for performing chained automated reasoning. 

Several existing resources provide semantic representations of general open-domain actions e.g., FrameNet, PropBank, and VerbNet. 
However, these resources do not adequately cover many of the scientific processes. 
FrameNet, for instance, does not have entries for nearly half of the processes described in 4th grade science exams. 
The coverage is likely to be worse for higher grade levels.

We propose to build a knowledge base about physical, chemical, and biological processes from their textual descriptions. 
The central idea is to automatically compose a semantic representation using a pre-determined vocabulary of semantic roles. 
Having identified the roles of interest, we will seek out sentences that express these roles and build extractors for these roles via bootstrapping.
%\footnote{We call these extractors rather than semantic parsers, since the goal here is to build knowledge about these processes and  not necessarily to build a parser that can reliably identify {\em all} semantic roles expressed in a sentence.} 
We will conduct intrinsic and external evaluations. 
We will create a manual target representation for intrinsic evaluation, and use the 4th grade science questions as an external application.

}